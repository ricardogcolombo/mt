\begin{algorithm}[H]
\caption{Clusterización de puntos cartesianos}
\begin{algorithmic}[1]
\Function{clusterizar}{Vector $points$} \Comment $\mathcal{O}(n^2)$
	\State Lista ejes $\gets construirEjes(points)$ \Comment $\mathcal{O}(n^2)$
	\State Grafo AGM $\gets calcularAGM(points)$ \Comment $\mathcal{O}(n + mlog(n))$
	\State prune_edges(AGM) \Comment $\mathcal{O}(m*(n + \triangle(G)))$
	\State Lista clusters $\gets findClusters(AGM)$ \Comment $\mathcal{O}(n)$
	\State \Return $\_clusters$
\EndFunction
\end{algorithmic}
\end{algorithm}

Donde findClusters es una función que simplemente recorre las componentes conexas con union disjoint set, como hace el algoritmo de Kruskal. 

\vskip 8pt